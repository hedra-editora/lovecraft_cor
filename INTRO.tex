\baselineskip=1.1\baselineskip

\chapterspecial{Introdução}{O espaço como horror}
%\addcontentsline{toc}{chapter}{Introdução, \textit{por Dirceu Villa}}

\begin{flushright}
\textsc{dirceu villa}
\end{flushright}

\noindent{}Das idiossincrasias do autor também surge este \textit{A cor que caiu do espaço}. Bastante consciente da divisão qualitativa naquilo que se pode chamar \textit{literatura fantástica}, Lovecraft assinala, em carta de 1934 a um Mr.\,Nelson --- já na última fase de sua escrita (viria a morrer em 1937) --- que uma coisa é Edgar Allan Poe, Ambrose Bierce (1842--1913) e Lord Dunsany (1878--1957), etc., autores do que chama ``auto"-expressão'', e outra aquilo que define como algo ``composto artificialmente para atender certas demandas do leitor superficial \& convencional''.

Enquadrava nessa segunda categoria a vulgaridade de muitos escritores de ficção
científica (\textit{pulp}, como era o caso do período) quando confrontados
com um problema singular: como representar entidades extraterrestres? A
resposta corriqueira --- e que continua a ser corriqueira, basta ver um
filme recente de muito sucesso, \textit{Arrival} ou \textit{A chegada}, de 2016,
dirigido por Denis Villeneuve\footnote{No filme, as criaturas têm a forma lovecraftiana de um dos Antigos, e parecem octópodes, com longos tentáculos que até mesmo expelem tinta negra. Quase nada escapa dessa regra. Nos filmes, excetuam"-se apenas alguns, como \textit{Invasion of the Body Snatchers} (ou \textit{Os invasores de corpos}, de Philip Kaufman, 1978) que teve um antecedente em 1956 e vem da história \textit{The Body Snatchers}, de Jack Finney, serializada pela Colliers Magazine em 1954, ou \textit{The Thing} (ou \textit{O enigma de outro mundo}, de John Carpenter, 1982) também um tipo de refilmagem de \textit{The Thing from Another World}, de 1951, e que tem igualmente como base um \textit{pulp} de 1938, \textit{Who Goes There?}, escrito por John W.\,Campbell Jr. e publicado pela revista \textit{Astounding Tales}. Lovecraft, em 1927, antecede a ambas, e podemos dizer que, enquanto os \textit{invasores de corpos} assimilavam"-se inicialmente a formas como que de plantas, e a \textit{coisa} era também um tipo de parasita alienígena que assumia corpos terrestres e se reconfigurava neles, os alienígenas de Lovecraft são muito mais imaginativos porque entidades indefiníveis. Lovecraft supera o esquema alegórico dessas outras ótimas narrativas, porque não propõe uma visão a ser lida como índice para outra coisa na realidade: o conto é, em si mesmo, a proposição.} --- é fazê"-los
humanoides, ou o resultado de algum tipo de evolução alternativa a
partir de criaturas encontráveis em nossa natureza terrestre.

Não seria apenas um caso flagrante de falta de imaginação --- ou de
restrições acabrunhantes de um referencial composto apenas por
experiências deste planeta e da percepção humana ---, mas uma falha
elementar de \textit{filosofia}: seria mais inteligente supor que algo que
viesse de fora da nossa experiência, tanto física como conceitual, seria
muito provavelmente tão distante daquilo que conhecemos e
\textit{reconhecemos} como vida, que provocaria terror já pela total
surpresa e pelo esforço de integrarmos aquilo a qualquer coisa redutível
ao conhecido.

Um exemplo de como costumamos proceder diante do desconhecido: na
\textit{Naturalis Historia} ou \textit{História natural} de Plínio, o velho (24--79
d.C.), é muito interessante observar que, nas descrições de animais e
plantas com que não teve contato visual, o naturalista compõe então uma
criatura arranjada de partes daquilo que é conhecido, resultando em um
tipo de híbrido \textit{imaginário} de algo \textit{real}, e daí temos
mariposas que sugam sangue, temos as \textit{lacrimis arborum}, as \textit{lágrimas
de árvores},\footnote{Plínio, o velho, \textit{Histoire Naturelle}. Paris: Société d'Édition Les Belles Letters, 1947, p.\,33.} as abelhas que
sofrem de uma tristeza paralisante, o uso do canino direito do lobo em
operações mágicas,\footnote{\textit{Idem}, p.\,81.} ou, supondo que, pela
proximidade da saliva e do sangue nos felinos, mesmo os domésticos, eles
seriam \textit{inuitat ad rabiem} ou \textit{inclinados à
ferocidade}.\footnote{\textit{Ibidem}, p.\,83.} O rinoceronte, por
exemplo, tem um caminho desses na história da aproximação humana com
suas peculiaridades. Plínio o descreve bastante fielmente:

\begin{quote}
um animal que tem um chifre só, projetando"-se do focinho; tem sido visto
frequentemente desde então. Também ele é um inimigo natural do elefante.
Prepara"-se para o combate afiando seu chifre contra as rochas; e ao
lutar dirige"-o sobretudo contra a barriga de seu adversário, pois sabe
que é a parte mais sensível. Os dois animais são de comprimento igual,
mas as pernas do rinoceronte são muito mais curtas: sua pele tem a cor
de madeira"-de"-buxeiro.\footnote{\textit{The Natural History of Pliny the
  Elder}. Londres: Henry G.\,Bohn, 1858, livro \textsc{viii}, capítulo \textsc{xxxix}.}
\end{quote}

Afirma, logo depois, que um de seus grandes inimigos naturais (além do
elefante) é o dragão, e as notas recentes ao texto supõem que com
\textit{dragão} Plínio referia algum tipo de serpente, pelo motivo talvez
decepcionante, mas até razoável, de que dragões não terão existido.

A respeito de elefantes, seria ainda curioso fazer notar como aquelas
compilações da autoridade antiga em história natural, os bestiários,
procediam seus catálogos descritivos de fauna. Em particular, um
bestiário do século \textsc{xiii}, da Bodleyan Library de Oxford, propõe a tromba
do elefante da seguinte maneira: ``Seu nariz é chamado tromba porque ele
o utiliza para levar comida à boca; a tromba é como uma cobra, e é
protegida por uma fortificação de marfim'',\footnote{\textit{Bestiary}. Londres: The Folio Society, 1992, p.\,40.} no qual explicitamente
recorre a ousadas combinações de imagens para tornar visível seu animal
fantástico (e real) para quem lê.

A história prosseguiria na xilogravura de Albrecht Dürer (1471--1528),
representando, em 1515, um rinoceronte do qual tinha apenas uma
descrição de alguém que vira um exemplar indiano do perissodáctilo,
enviado como um presente do papa Leão \textsc{x} a Lisboa, para o rei Manuel\,\textsc{i}. E
então Dürer o desenha como que com chapas de armadura revestindo o corpo
colossal, fazendo"-o em parte verdadeiro, em parte imaginário; e, no
século \textsc{xx}, Salvador Dalí (1904--1989) o utilizaria, já com pleno
conhecimento do que é um rinoceronte em 1954, para representá"-lo com
rendas cobrindo sua bizarríssima couraça.

Tudo o que está acima são exercícios de figurar o desconhecido de modo a
torná"-lo familiar, passível de composição, imaginável; ou, por outro
lado, com o objetivo de distorcer o real para fazê"-lo comportar a
dimensão mais ampla da hipótese.

O que pretendo dizer com esse excurso da história natural é que
Lovecraft é o primeiro de que tenho notícia a romper com esse sistema de
aproximação do desconhecido: em \textit{A cor que caiu do espaço}, o que faz
com engenho incomparável é propor um tipo de forma de vida indefinível
sob nossos padrões de observação, algo que, para os personagens humanos
que entram em contato com ela (e os leitores), é um horror completo e
quase uma abstração. Fazê"-lo exigiu algumas referências, uma curiosidade
muito inteligente e o labor intenso para escapar daquilo que poderia, de
outra forma, ser apenas um truque barato para manipular os medos de seu
público. Lovecraft fez, portanto, muito mais: nos fez uma proposição
hipotética, pretendeu expandir a percepção.

\section*{A invenção do incomunicável}

\textit{The Colour Out of Space} é um conto publicado em setembro de 1927
pela revista \textit{pulp}\footnote{Para uma síntese da história das
  revistas e do tipo de narrativa \textit{pulp}, ver p.\,\pageref{pulp}.} Amazing Stories. Lovecraft o escreveu quando
trabalhava no ensaio \textit{Supernatural Horror in Literature} ou \textit{Horror
sobrenatural em literatura}, e interessa considerar o resultado de sua
exploração intelectual do tema na narrativa notável que produz, geminada
ao ensaio, porque é bastante claro que algumas questões do ensaio --- em
especial aquelas nas quais Lovecraft efetivamente abria novos rimos para
a ficção especulativa --- influíram na escrita desse texto, ou ganharam
nele sua demonstração ficcional.

Na opinião do próprio Lovecraft, esse conto era uma de suas narrativas
favoritas, e não por acaso: terá percebido que concebera algo singular,
algo que ninguém antes fora capaz de registrar, e que, acrescento,
ninguém mais registrou tão bem em ficção. Se chamamos esse texto
\textit{ficção científica}, nós o empobrecemos, não porque ficção
científica seja algo menor, mas porque nele há mais do que uma hipótese
científica, ou uma alegoria do presente lançada em um futuro
prospectivo. Poderíamos dizer que é tanto científico quanto filosófico,
além de, naturalmente, ser uma das histórias mais horripilantes já
concebidas.\footnote{\textit{Colour Out of Space}, de 2019, o filme recente,
  em parte consegue se aproximar do aspecto chocante, por exemplo, na
  cena do sótão (a cena do celeiro é quase uma citação a \textit{The
  Thing}, filme de 1982 dirigido por John Carpenter, que, por sua vez,
  buscava ser lovecraftiano), mas falha em instilar o permanente
  desassossego, e perde muito da sutileza engenhosa de construção de
  atmosfera.}

Um dos pontos fundamentais para a consideração do que se passa no conto
tem uma âncora factual igualmente extraordinária, o caso das ``garotas
radiativas'': jovens que haviam conquistado empregos bem pagos para
executar pinturas pequenas, de precisão, em diversos produtos (como os
aros dos mostradores de relógios), utilizando um material então na moda,
o elemento \textit{rádio} --- brilhante por sua radio luminescência --- em
seus pincéis, que elas lambiam para dar às cerdas melhor ponta, como
instruíam seus superiores propondo o lema \textit{lip, dip, paint}.\footnote{Em tradução para o português, \textit{lambe,mergulha, pinta}.}

Elas não sabiam ainda da alta toxicidade do rádio, e acabavam seus
turnos com o rosto e o cabelo cintilantes, quando saíam à rua, o que
causava fascínio não apenas nelas, e chegou mesmo a significar certo
\textit{status} social. Mas, pouco depois, a ação da radiatividade fez com
que começassem a desenvolver cânceres, que surgiam antes na mandíbula (a
chamada ``mandíbula radiativa'') e chegavam até os ossos, passando
espantosamente a emitir radiação de dentro para fora.\looseness=-1

Sofreram preconceito de toda parte, quando mesmo colegas não acreditavam
em suas versões de que o rádio as estava matando, e supunham que morriam
de sífilis: uma condenação moral de puro preconceito. Os processos
contra as companhias empregadoras demonstraram que eram expostas a um
perigo potencial desconhecido para elas, mas descobriu"-se que os
técnicos já se protegiam durante a manipulação. Seu caso, aviltante e
demorado, acabou servindo para fazer avançar juridicamente os direitos
trabalhistas sobre as doenças ocupacionais, e para expor o perigo de um
material então comercializado sem precaução.

Mas o caso também teria, como se supõe, alimentado a imaginação atenta
de Lovecraft: os ossos se fraturavam e decaíam facilmente, a necrose se
instalava nas mandíbulas, o que se iniciava com sangramento das gengivas
e acabava com o desfigurar do rosto por tumores e ossatura porosa, como
descrito e diagnosticado por um dentista ainda em 1924. Essa força
invisível, que ao mesmo tempo operava modificações completas no corpo
--- que perdia as cores, também --- até levar à morte, sugeria não
apenas mais uma hipótese sorrateira do desconhecido, mas renovava aquele
princípio frankensteiniano dos limites da ciência, quando é impossível
determinar a natureza e o efeito do que há num universo vasto demais
para ser apreendido.

Podemos notar distintamente o efeito dessa história macabra --- colhida
no mundo que apenas adentrava o uso arriscadíssimo e logo militarizado
da radiação --- em ``A cor que caiu do espaço'', no qual todas as
características horríveis da ação radiativa se observam, incluindo a
devastação de lugares que se tornam desertos de poeira cinzenta; mas
seria também necessário lembrar que não se trata nem precisamente de
radiação, e ainda menos de um \textit{ataque} extraterrestre: sequer se
define algum objetivo discernível no efeito daquilo que se sentia como
uma presença. A desproporção do evento é a desproporção entre os tipos
de entidades confrontados pelo incidente.

Mas Lovecraft ainda teria outros pontos de referência para sua
composição.

\section*{As pedras de raio}

Em \textit{The Colour Out of Space} tem"-se, portanto, uma forma de vida mal
percebida, seja por pequenos fazendeiros locais ou por experimentados
cientistas, que sequer poderiam estar certos de que o que presenciavam
era de fato uma forma de vida.

A cor do título é uma sagacidade inteiramente literária, por ser
possível apenas em literatura explorar algo que, do ponto de vista dos
sentidos (e do próprio intelecto), é intangível. Lovecraft desenvolve
seu artesanato numa linha impossível às outras artes, basta ver que, na
adaptação de 2019 de seu conto para o cinema --- \textit{Colour Out of
Space}, de Richard Stanley, com Nicolas Cage ---, a cor indefinível do
conto, inexistente no nosso planeta, é convertida, pela necessidade de
representação visual, em \textit{magenta} ou \textit{fúcsia}, porque tal cor
não existe na natureza em uma única faixa de ondas do nosso espectro
visual. Conquanto o engenho se justifique, a eficácia de sua invenção é
insignificante se posta perto de sua origem literária lovecraftiana:
lendo o conto se percebe entre os personagens o desconcerto completo, a
perplexidade diante da cor inexistente, indescritível, quase nem cor,
que já por isso interfere modificando a consciência de quem entra em
contato com a estranheza absoluta.

E esse é outro ponto fundamental da história: Lovecraft implica uma
ambivalência de aspecto sobrenatural na alteração de estrutura de tudo
aquilo com que a emanação alienígena entra em contato. Pode"-se entender
a coisa de modo próprio ou figurado: tanto pode ser algo que altere
fundamentalmente as relações do novo ambiente, por condições
físico"-químicas inteiramente estranhas para este mundo, quanto pode ser
a proposição de que alterar a percepção e a consciência significa
alterar a própria realidade. Assim, nada mais funcionaria: surgem
frequências de ondas incompatíveis com o espectro visível, distorções de
toda espécie, auditivas, tácteis, irregularidades genéticas, abominações
físicas revertendo toda ordem conhecida, afetando de modo catastrófico
percepção, intelecção e biologia.\looseness=-1

Essa proposição ficcional arregimenta grandes forças também por outro
motivo: como propus acima, Lovecraft não o faz pelo simples motivo de
compor engenhosa fantasia, mas pretende seriamente desafiar, com as
noções especuladas em seu conto, o que descreve em seu texto sobre
\textit{Horror sobrenatural em literatura} como o modo desdenhoso da
``sofisticação materialista'', apegada aos ``eventos externos'', que
constrói uma narrativa didática com algum nível de ``otimismo
sorridente'', ao que ele contrapõe ``as visões amplificadas'' das
novidades incipientes da ciência, como ``química intra"-atômica,
astrofísica avançada, doutrinas da relatividade, e sondagens da biologia
e do pensamento humano''.

É notável em seus dois textos esse aspecto duplo, feito de \textit{ficção}
e de \textit{convicção}: sua escrita tem, guardadas as proporções, o
fervor daquele antigo dispositivo, a visão, que encontramos em poetas
visionários como William Blake ou Arthur Rimbaud. Lovecraft é sério a
respeito do que compõe, não pretende apenas oferecer uma distração
moderadamente estimulante, mas está apresentando, na verdade, um arranjo
prospectivo e perceptivo, o que deveria ser o princípio estruturante de
toda a chamada \textit{literatura especulativa}.

Nesse ponto interessa considerar ainda outro influxo que serviu para dar
base a \textit{A cor que caiu do espaço}, o livro de Charles Fort
(1874--1932), \textit{Book of the Damned} ou \textit{Livro dos Malditos}, de 1919, que,
apesar do título espetacular, busca na verdade combater um tipo de visão
convencional que Fort acreditava estar tornando a ciência em uma espécie
de religião por excluir dados e informações que não pudessem ser
atendidos pelo sistema científico. Os \textit{malditos}, assim, são os dados
e as informações que reúne no livro, referências, relatos, fatos e
histórias que constituiriam um grupo de fenômenos ou incidentes,
ignorados ou minimizados em suas singularidades, por uma ciência que,
segundo Fort, vinha se tornando um bloco de concordância tão sólido
quanto qualquer fé.

Em um ponto encontramos algo que terá chamado a atenção de
Lovecraft\footnote{Lovecraft não apenas é associado a Fort por pontos de
  conexão do imaginário de ambos, mas também por tê"-lo citado em \textit{The
  Whisperer in Darkness} (1931).} naquele livro peculiaríssimo, o ponto
onde Fort discorre sobre as \textit{thunderstones}, as \textit{pedras de raio} ou
\textit{pedras de Thor}, que folcloricamente tinham o poder de atrair raios,
e muitas delas tinham origem em meteoritos. Quando Fort, em busca de
padrões, está registrando ocorrências de quedas de aerólitos durante
tempestades, alterações nas formas das nuvens, e muito do que é
atribuído à ``ignorância do camponês'', ele escreve:

\begin{quote}
Ou --- estaríamos a caminho de explicar as \textit{pedras de raio}. Parece"-me
que, particularmente notável, se confirma a aceitação geral de que a
nossa não passa de uma existência intermediária, na qual nada é
fundamental, ou nada é definitivo para se tomar como um padrão positivo
a partir do qual julgar. Camponeses acreditam em meteoritos. Cientistas
excluem os meteoritos. Camponeses acreditam em \textit{pedras de raio}.
Cientistas excluem as \textit{pedras de raio}. É inútil argumentar que os
camponeses estão lá no campo, e os cientistas estão fechados em
laboratórios e salas de conferência. Não podemos tomar como base real
que, quanto aos fenômenos com os quais têm maior familiaridade,
camponeses têm mais probabilidade de estar certos do que os cientistas:
uma pletora de falácias biológicas e meteorológicas dos camponeses se
ergue contra nós.\footnote{Charles Fort, \textit{Book of the Damned}. San
  Diego: The Book Tree, 2006, pp.\,75--6.}
\end{quote}

Cito apenas o trecho que me parece mais contundente --- se comparado com
a narrativa de Lovecraft ---, mas há ainda tantas outras instâncias nas
quais o autor de \textit{A cor que caiu do espaço} poderia ter achado
interesse vivo. Nesse caso que mencionei, porque no texto de Fort se
põem em contraste os modos habituais de se considerar ``a ignorância
do camponês'' e ``o conhecimento do cientista'', o que nos leva de
imediato à cena, no conto de Lovecraft, da queda do estranho meteorito
nas terras de Nahum Gardner, o camponês e sua família que receberão as
visitas de cientistas da Universidade Miskatonic, em Arkham, cidade
grande e próxima; os mesmos cientistas que desistirão de tentar entender
o fenômeno, diante das frustrantes condições, aliás indescritíveis,
daquele mineral extraterreno, e dos \textit{causos} de interior que ouvem de
pessoas simples.

Mas os efeitos daquele objeto prosseguirão modificando os camponeses em
contato com ele, e, assim, há uma clara oposição entre aqueles que estão
``no campo'', tendo de lidar com os eventos diretamente, e os que estão
em ``laboratórios e salas de conferência'', e podem deixar o assunto
assim que sentirem que não irão avançar no entendimento. Sente"-se a
hipótese do trecho de Fort sendo desenvolvida ficcionalmente pela
imaginação de Lovecraft, combinando as notícias aterradoras das garotas
radiativas e seus cânceres deformantes (pelo elemento radiativo e sua
luminescência), e as estranhas ocorrências meteoríticas levantadas sob o
título momentoso, \textit{maldito}, de Fort.

S.\,T.\,Joshi, estudioso e mais importante biógrafo de Lovecraft, escreve
em seu livro \textit{A Subtler Magick: The Writings and Philosophy of H.\,P.\,Lovecraft}, de 1996, acrescentando um detalhe que também situa, com outro dado histórico, o
momento de criação do conto: Joshi menciona o Quabbin Reservoir,
reservatório cuja construção foi anunciada em 1926 (embora só ficasse
pronto em 1939) e que ``se localiza exatamente na área central de
Massachusetts, onde o conto transcorre, e que envolvia o abandono e a
submersão de povoados inteiros na região'',\footnote{S.\,T.\,Joshi, \textit{A
  Subtler Magick: The Writings and Philosophy of H.\,P.\,Lovecraft}.
  Gillette: Wildside Press, 1996, p.\,135.} precisamente como acontece na
narrativa, também com a visita do narrador, o perito que está
supervisionando a região antes da chegada das máquinas.

Talvez por isso, quero dizer, pelo material extensivo tomado de fatos à
volta de Lovecraft --- reunidos por uma concentração inventiva que não
só os animou, mas também pôde propor uma questão filosófica de
representação, que vimos ocupar sua inteligência com a redação do ensaio
sobre o horror sobrenatural na ficção --- \textit{A cor que caiu do espaço}
tenha se tornado o conto por excelência da marca mais significativa que
Lovecraft deixou. Visão ainda hoje sem rivais, a construção da narrativa
diz a que veio já no primeiro parágrafo, em um artesanato delicado de
linguagem compondo visualmente o lugar em nossas imaginações: é claro,
Lovecraft sabia que urdia algo especial.

\section*{Um pouco de linguagem, e uma paródia}

Se a linguagem urdida do primeiro parágrafo começa a nos introduzir numa
atmosfera, cuja invenção é fundamental para poder ir extraindo seus
efeitos na leitura, será possível também perceber que Lovecraft
pretendeu, da mesma maneira, aplicar"-se na confecção de ainda outro tipo
específico de abordagem linguística.

Há algo como uma paródia desse famoso conto no filme que juntou Stephen
King e o grande diretor de \textit{Night of the Living Dead} ou \textit{A noite dos
mortos"-vivos}, de 1968, George Romero: \textit{Creepshow}, de 1982, é um
cultuado filme de terror, estruturado em vários episódios que contam
histórias diversas e independentes, como uma \textsc{hq}. No episódio ``A morte solitária de Jordy Verill'',\footnote{No original, ``The Lonesome Death of Jordy Verill''.}
camponês caricato (o próprio Stephen King) que vive sozinho em uma casa
rústica, vê a queda de um meteorito em sua propriedade. O meteorito se
abre e um brilho verde radiativo se expande dele: ao tocá"-lo, o pobre
Verill começa a ter o corpo tomado de espessa vegetação, como tudo a seu
redor.

King certamente pensava em Lovecraft, seu conterrâneo da Nova Inglaterra
(King é originário de Portland, no Maine) e escritor admirado, ao
escrever o conto \textit{Weeds}, de 1976, conto que depois resultaria no
episódio paródico de \textit{Creepshow}. Há mesmo a caricatura extrema da
linguagem interiorana de seu personagem, o que também é uma piscadela
para quem tenha lido \textit{A cor que caiu do espaço} e notado o cuidado com
que Lovecraft registra sua hipótese da fala simples de Nahum Gardner (e
de Ammi Pierce) quando finalmente se lhe dá voz, já perto do fim. Lá
também se nota o empenho da escrita lovecraftiana em compor aquele
triste momento, no qual o pobre homem tenta, sem muito sucesso,
comunicar a seu amigo algo da experiência indefinível que sua família,
seus animais, sua terra e ele mesmo enfrentaram.

É um conto singular, o favorito de Lovecraft, e o deste seu tradutor e
introdutor. Boa leitura.

