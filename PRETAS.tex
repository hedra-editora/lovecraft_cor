\textbf{H.\,P.\,Lovecraft} demonstrou desde a infância interesse pelas artes e pelas ciências, logo se tornando um leitor precoce. Aos dois anos, viu o pai ser internado em um manicômio, onde permaneceria até morrer cinco anos mais tarde. Lovecraft continuou morando com a família, porém a mãe jamais se recuperou da perda e começou a sofrer distúrbios mentais que afetaram profundamente sua relação com o filho, criando laços doentios entre os dois. Após uma crise nervosa em 1908, quando ainda estava em idade escolar, Lovecraft abandonou para sempre os estudos e passou a levar uma existência reclusa até que, em 1914, descobriu o jornalismo amador. A partir de então principiou a publicar contos de horror e variados artigos em diversos periódicos, bem como a dedicar"-se à vasta correspondência que manteria ao longo de toda a vida. Depois de perder a mãe em 1921 e de se casar em 1924, passou uma temporada de penúria em Nova York; esta, somada ao fracasso de seu casamento, obrigou"-o a voltar para a casa de suas tias em Providence. Morreu em 1937, vítima de câncer do intestino.

\textbf{A cor que caiu do espaço} é a obra mais representativa do momento decisivo de Lovecraft em direção ao horror cósmico inspirado pela ficção científica. Na história, um vilarejo a oeste de Arkham vê-se ameaçado quando um meteoro cai na propriedade de um fazendeiro local e traz consigo uma estranha aberração cromática que afeta a flora e a fauna da região --- e cria o cinzento e estéril “descampado maldito” onde nada cresce.

\pagebreak

\textbf{Dirceu Villa} é poeta, tradutor e ensaísta. Publicou cinco livros de poesia, \emph{\textsc{mcmxcviii}} (1998), \emph{Descort} (2003), \emph{Icterofagia} (2008), \emph{Transformador} (2014), \emph{speechless tribes: três séries de poemas incompreensíveis}, e, entre outros, traduziu \emph{Um anarquista e outros contos}, de Joseph Conrad (2009), \emph{Lustra}, de Ezra Pound (2011), \emph{Famosa na sua cabeça}, de Mairéad Byrne (2015) e \emph{O Anjo Heurtebise}, de Jean Cocteau (2020). Doutor em Literaturas de Língua Inglesa pela \textsc{usp} (com estágio de doutorado em Londres), estudando o Renascimento na Inglaterra e na Itália, e pós"-doutorado em Literatura Brasileira. Há sete anos é professor da Oficina de Tradução Poética da Casa Guilherme de Almeida (Centro de Estudos de Tradução Literária).

\blankpage



